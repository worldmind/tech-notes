\documentclass[aspectratio=169]{beamer}
\usepackage[orientation=landscape,size=custom,width=16,height=9,scale=0.5,debug]{beamerposter}
\usepackage[T2A]{fontenc}
\usepackage[utf8]{inputenc}
\usepackage[english,russian]{babel}
\usepackage{cite,enumerate,float,indentfirst}
\usepackage{listings}
\lstset{language=Perl}
\usetheme{umbc2}

\title{Unit testing for Perl}
\author{Alexey Shrub}
\institute{Российские интернет-технологии}
\date{2011-04-26}
\begin{document}

%%титульная страница
\maketitle

%% Что такое модульное тестирование
\begin{frame}{Модульное тестирование}
\begin{itemize}
\item Автоматизированное.
\item Изолированное.
\end{itemize}
\end{frame}

%% Зачем нужны тесты
\begin{frame}
\frametitle{Зачем нужны модульные тесты}
\begin{itemize}
\item Необходимая верификация (+ двойная запись).
\pause
\item Борьба с ростом энтропии (регрессом) при изменениях (= легкость рефакторинга).
\pause
\item Локализация ошибок (в отличие от интеграционных).
\pause
\item Раннее обнаружение ошибок (чем раньше, тем дешевле исправление ошибки).
\pause
\item Раннее обнаружение неудобного интерфейса.
\pause
\item Документация.
\end{itemize}
\end{frame}

%% Почему мало кто их пишет?
\begin{frame}
\frametitle{Стандартные отмазки нежелающих писать тесты}
\begin{itemize}
\item Нет времени.
\item Код нетестируемый.
\item Не умею и боюсь, у меня и без тестов вроде/должно работать.
\end{itemize}
\end{frame}

\begin{frame}
\begin{center}
Тесты в Perl. Функциональное тестирование
\end{center}
\end{frame}

%% Test::More
\begin{frame}[fragile]
\frametitle{use Test::More;}
Базовые функции
\begin{itemize}
\item ok
\item is
\item new\_ok
\item is\_deeply
\item ...
\end{itemize}
Диагностика (diag/explain):
\begin{lstlisting}
is_deeply( $got, $expected, 'Result must be ...' )
    or diag explain $got;
\end{lstlisting}
\end{frame}

%% Минимальный пример теста
\begin{frame}[fragile]
\frametitle{Минимальный пример}
\begin{block}{Пример положительного функционального теста}
\lstinputlisting{t/simple-test.t}
\end{block}
\end{frame}

%% Запуск теста
\begin{frame}[fragile]
\frametitle{Запуск одного теста}
TAP - Test Anything Protocol
\begin{block}{Run test}
\begin{verbatim}
$ perl t/simple-test.t
1..1
ok 1 - world.mind@yahoo.com must be valid
\end{verbatim}
\end{block}
\end{frame}

%% Запуск набора тестов
\begin{frame}[fragile]
\frametitle{Запуск набора тестов}
\begin{block}{Run tests with Test:Harness}
\begin{verbatim}
$ prove
t/simple-test.t .. ok   
t/use.t .......... ok   
All tests successful.
Files=2, Tests=2,  1 wallclock secs ( 0.02 usr  0.01 sys +  0.14 cusr  0.02 csys =  0.19 CPU)
Result: PASS
\end{verbatim}
\end{block}
Makefile - бывает удобнее
\end{frame}

%% Тестирование исключений
\begin{frame}{Тестирование исключений}
\begin{block}{Test::Exception}
\lstinputlisting{t/exception.t}
\end{block}
\end{frame}

%% 
\begin{frame}{Генерация входных данных}
\begin{block}{Test::LectroTest::Compat}
\lstinputlisting{t/lectro.t}
\end{block}
\end{frame}

%% Модуль взаимодействует с внешними объектами?
\begin{frame}{Что делать, если модуль взаимодействует с внешним миром?}
\begin{itemize}
\item Пишет/читает базу.
\item Обращается к web страницам/скриптам.
\item Пишет/читает memcache.
\item Вызывает SOAP/XML-RPC сервисы.
\item и т.п.
\end{itemize}
\begin{center} 
\LARGE ?
\end{center}
\end{frame}

%% Mock/Stub/Fake
\begin{frame}{Mock/Stub/Fake}
Mock модули общего назначения
\begin{itemize}
\item Test::MockObject
\item Test::MockModule
\item Test::MockClass
\end{itemize}
Специализированные
\begin{itemize}
\item DBD::Mock
\item Test::Mock::LWP
\item Cache::Memcached::Mock
\item и т.п.
\end{itemize}
\end{frame}

%% Пример подмены LWP
\begin{frame}[allowframebreaks]{Пример подмены модуля LWP}
\lstinputlisting{t/mock-memcache.t}
\end{frame}

%% Закончили с функциональным
\begin{frame}{}
\begin{center}
Нефункциональное тестирование

Автоматизированный code review

Почему?

Зачем?

\end{center}
\end{frame}

%% 
\begin{frame}{Компилируется?}
\begin{itemize}
\item Test::Strict
\end{itemize}
\lstinputlisting{t/strict.t}
\end{frame}

%% 
\begin{frame}{Соответствует соглашению о стиле кодирования?}
\begin{itemize}
\item Test::EOL
\item Test::NoTabs
\item Test::PerlTidy
\end{itemize}
\end{frame}

%% 
\begin{frame}{Используются ли рекомендации из Perl Best Practice}
\begin{itemize}
\item Test::Perl::Critic
\item Test::Portability::Files
\end{itemize}
\end{frame}

%% 
\begin{frame}{Не забыли ли чего? (инструменты в больном)}
\begin{itemize}
\item Test::Fixme
\item Test::NoBreakpoints
\end{itemize}
\end{frame}

%% 
\begin{frame}{Метрики в норме?}
\begin{itemize}
\item Perl::Metrics::Simple
\end{itemize}
\end{frame}

%% 
\begin{frame}{Есть ли документация?}
\begin{itemize}
\item Test::Pod
\item Test::Pod::Coverage
\item Test::Spelling
\end{itemize}
\end{frame}

%% 
\begin{frame}{Есть ли нужное количество тестов?}
\begin{itemize}
\item Test::Strict (Devel::Cover)
\end{itemize}
\end{frame}

%% 
\begin{frame}{Не стал ли код медленнее?}
\begin{itemize}
\item Test::Timer
\end{itemize}
\end{frame}

%% 
\begin{frame}{Нет ли утечек памяти?}
\begin{itemize}
\item Test::Weaken
\end{itemize}
\end{frame}

%% 
\begin{frame}{О чём говорит успешное прохождение таких тестов?}
\begin{itemize}
\item Код компилируется! Это уже успех!
\item Стиль кодирования соответствует заданному!
\item Выполняются хотя бы минимальные рекомендации из PBP!
\item Доделано всё, о чем были пометки!
\item Метрики сложности дают надежду на то, что код можно понять!
\item Была попытка написать документацию ко всем методам!
\item Есть тесты! И их количество соответствует запланированному!
\item Код ещё не самый тормозной!
\item Можно надеяться на то, что память не течёт!
\end{itemize}
\end{frame}

%% 
\begin{frame}{}
Максимальный набор, все кроме последних двух не зависят от кода, можно копипастить и запускать
\end{frame}

%% 
\begin{frame}{Вопросы}

\begin{center}
\LARGE QUESTIONS?
\end{center}

\begin{block}{Исходники презентации (LaTeX, Beamer):}
https://github.com/worldmind/perl-unit-testing-presentation-ru.git
\end{block}

\begin{block}{Набор тестов:}
https://github.com/worldmind/perl-test-code-quality-template.git
\end{block}

\begin{block}{Feedback to:}
ashrub@yandex.ru
\end{block}

\end{frame}

\end {document}

