\documentclass[aspectratio=169]{beamer}
\usepackage[orientation=landscape,size=custom,width=16,height=9,scale=0.5,debug]{beamerposter}
\usepackage[T2A]{fontenc}
\usepackage[utf8]{inputenc}
\usepackage[english,russian]{babel}
\usepackage{cite,enumerate,float,indentfirst}
\usepackage{graphicx}
%\usetheme{umbc2}

\title{Haskell это круто!}
\author{Alexey Shrub}
\date{2015-09-29}
\begin{document}

\maketitle

\begin{frame}{Математика - лучший костыль для разума}
\begin{itemize}
\item Математика - фундамент успехов науки и технологий
\pause
\item Новые технологии одни проблемы решают, другие создают - упускаем что-то, 7+-2?
\pause
\item Гарантированно безопасное автоизменение кода (лень, параллель, оптимизация)
\pause
\item Доказательство программ
\pause
\item "Это вовсе не значит, что одна причина лучше двух; просто если вы предлагаете себе больше одной причины, значит, вы пытаетесь в чем-то себя убедить. Очевидные решения (неуязвимые в отношении ошибок) требуют не больше одной причины" Нассим Талеб
\end{itemize}

\end{frame}

\begin{frame}{Высокий уровень абстракции}
\begin{itemize}
\item Высокий уровень = эффективность разработки (МакКоннел)
\pause
\item Превращается в DSL (see on youtu.be: Programming - Why Haskell is Great)
\pause
\item Число строк в Warp (111188/12486/5270, ~500)
\pause
\item Выше уровень - умнее компилятор (подкапотный неблокирующий IO, управление памятью)
\end{itemize}
\end{frame}

\begin{frame}{Подозрительный тип}
\begin{itemize}
\item Типы не хороши в прототипах, но хороши в продакшене
\pause
\item Чистота функций
\end{itemize}
\end{frame}

\begin{frame}{Просто круто}
\begin{itemize}
\item Суперпроффессиональное сообщество ( acid-state: "get stronger ACID guarantees than most RDBMS offer"!)
\pause
\item Реляционная алгебра для работы с СУБД с валидацие типов на стадии компиляции (HaskellDB)
\pause
\item warp vs nginx
\pause
\item Транзакционная память и много других плюшек которые в других языках ещё только изобретают ИМХО (см. Пол Грэм: «Месть ботанов» о Лиспе)
\end{itemize}
\end{frame}

\end {document}

